% Created 2020-04-24 vie 23:03
% Intended LaTeX compiler: pdflatex
\documentclass[12pt,a4paper, twosite]{article}
\usepackage[utf8]{inputenc}
\usepackage[T1]{fontenc}
\usepackage{graphicx}
\usepackage{grffile}
\usepackage{longtable}
\usepackage{wrapfig}
\usepackage{rotating}
\usepackage[normalem]{ulem}
\usepackage{amsmath}
\usepackage{textcomp}
\usepackage{amssymb}
\usepackage{capt-of}
\usepackage{hyperref}
\usepackage[left=2.00cm, right=2.50cm, top=2.50cm, bottom=2.00cm]{geometry}
\usepackage{fancyhdr}
\usepackage{tabularx}
\usepackage{graphicx}
\fancyhead[RO,LE]{\thepage}
\fancyhead[LO]{\emph{\uppercase{\leftmark}}}
\fancyfoot{}
\renewcommand{\headrulewidth}{1.0pt}
\pagestyle{fancy}
\date{}
\title{Master test plan para analizador modular de ambiente}
\hypersetup{
    pdfauthor={},
    pdftitle={IEEE-830},
    pdfkeywords={},
    pdfsubject={},
    pdfcreator={Emacs 26.2 (Org mode 9.1.9)},
    pdflang={English}}
\begin{document}

\maketitle
\tableofcontents

\newpage

\section*{Registros de cambios}
\label{sec:registro}

\begin{table}[ht]
    \label{tab:registro}
    \centering
    \begin{tabularx}{\linewidth}{@{}|c|X|c|@{}}
        \hline
        \bf Revisión & \multicolumn{1}{c|}{\bf Detalles de los cambios
        realizados}  &
        \bf Fecha
        \\ \hline
        V1.0         & Creación del documento
                     & 04/06/2022
        \\
        \hline
    \end{tabularx}
\end{table}

\pagebreak

\section{Introducción}
En el presente documento se detallarán todos los aspectos relacionados con la
especificación del Master Test Plan referentes al desarrollo del analizador
modular de ambiente, cuyo propósito es realizar una valoración del ambiente con
niveles de temperatura, gases presentes y ruido. Todos estos datos serán
visualizados por el usuario mediante una aplicación para smartphones.
\subsection{Contenidos}
Los contenidos del Master Test Plan son:
\begin{itemize}
    \item Asignaciones.
    \item Bases del test.
    \item Estrategia general del test.
    \item Estrategia por nivel de prueba
\end{itemize}

\section{Asignaciones}

\subsection{Responsable}
El responsable de la elaboración del documento es Luis Storaccio, encargado del
desarrollo del proyecto.
\subsection{Contratista}
La asignación es ejecutada bajo la responsabilidad de Luis Storaccio, jefe del
testing del desarrollo del proyecto.
\subsection{Alcance}
El alcance del test de aceptación es el dispositivo analizador modular de
ambiente.
\subsection{Objetivos}
Los objetivos son:
\begin{itemize}
    \item Determinar si el sistema cumple con los requerimientos.
    \item Reportar las diferencias entre comportamiento deseado y observado.
    \item Proveer de funciones de test de forma automática para el firmware, que
          pueda ser reutilizada en versiones futuras.
\end{itemize}
\subsection{Precondiciones}
Se realizarán formaciones para las pruebas que lo requieran.
\section{Bases del test}
Las bases del test consisten en:
\begin{itemize}
    \item Planificación y especificaciones del dispositivo.
    \item Especificación de requerimientos del software.
\end{itemize}
\section{Estrategia general del test}
\subsection{Características de calidad}
La siguiente tabla muestra la selección de las características de calidad
ISO/IEC9126 con su respectiva importancia relativa.

\begin{table}[ht]
    \centering
    \begin{tabular}{ll}\hline \hline
        \bf Características de calidad & \bf Importancia relativa (\%) \\ \hline
        \hline
        Funcionalidad                  & 20                            \\
        Confiabilidad                  & 30                            \\
        Usabilidad                     & 30                            \\
        Eficiencia                     &                               \\
        Mantenibilidad                 & 20                            \\
        Portabilidad                   &                               \\ \hline
    \end{tabular}
\end{table}
\begin{itemize}
    \item Funcionalidad: el software del producto debe cumplir con los
          requerimientos establecidos y es por ello que posee una importancia relativa
          del 20\%.

    \item Confiabilidad: es prescindible que el software presente un nivel de
          confiabilidad medio, debido a que el mismo realiza la lectura de valores de
          gases presente en el ambiente.
    \item Usabilidad:
    \item Mantenibilidad:
\end{itemize}

\end{document}