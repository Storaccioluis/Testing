\documentclass[
11pt, % The default document font size, options: 10pt, 11pt, 12pt
%codirector, % Uncomment to add a codirector to the title page
]{charter} 
\usepackage{pdfpages}
\usepackage{placeins}
\usepackage{graphicx}


% El títulos de la memoria, se usa en la carátula y se puede usar el cualquier lugar del documento con el comando \ttitle
\titulo{Calificador modular de ambiente} 

% Nombre del posgrado, se usa en la carátula y se puede usar el cualquier lugar del documento con el comando \degreename
\posgrado{Carrera de Especialización en Sistemas Embebidos} 
%\posgrado{Carrera de Especialización en Internet de las Cosas} 
%\posgrado{Carrera de Especialización en Intelegencia Artificial}
%\posgrado{Maestría en Sistemas Embebidos} 
%\posgrado{Maestría en Internet de las cosas}

% Tu nombre, se puede usar el cualquier lugar del documento con el comando \authorname
\autor{Luis Sebastián Storaccio} 

% El nombre del director y co-director, se puede usar el cualquier lugar del documento con el comando \supname y \cosupname y \pertesupname y \pertecosupname
\director{Ing. Matías Brignone}
\pertenenciaDirector{CESE-FIUBA} 
% FIXME:NO IMPLEMENTADO EL CODIRECTOR ni su pertenencia
%\codirector{John Doe} % para que aparezca en la portada se debe descomentar la opción codirector en el documentclass
\pertenenciaCoDirector{FIUBA}

% Nombre del cliente, quien va a aprobar los resultados del proyecto, se puede usar con el comando \clientename y \empclientename
\cliente{Luis Sebastián Storaccio}
\empresaCliente{Emprendimiento personal}

% Nombre y pertenencia de los jurados, se pueden usar el cualquier lugar del documento con el comando \jurunoname, \jurdosname y \jurtresname y \perteunoname, \pertedosname y \pertetresname.
\juradoUno{Nombre y Apellido (1)}
\pertenenciaJurUno{pertenencia (1)} 
\juradoDos{Nombre y Apellido (2)}
\pertenenciaJurDos{pertenencia (2)}
\juradoTres{Nombre y Apellido (3)}
\pertenenciaJurTres{pertenencia (3)}
 
\fechaINICIO{21 de octubre de 2021}		%Fecha de inicio de la cursada de GdP \fechaInicioName
\fechaFINALPlan{9 de diciembre de 2021} 	%Fecha de final de cursada de GdP
\fechaFINALTrabajo{17 de octubre de 2022}	%Fecha de defensa pública del trabajo final


\begin{document}

\maketitle
\thispagestyle{empty}
\pagebreak


\thispagestyle{empty}
{\setlength{\parskip}{0pt}
\tableofcontents{}
}
\pagebreak


\section*{Registros de cambios}
\label{sec:registro}

\begin{table}[ht]
    \label{tab:registro}
    \centering
    \begin{tabularx}{\linewidth}{@{}|c|X|c|@{}}
        \hline
        \rowcolor[HTML]{d6c6c3} 
        \bf Revisión & \multicolumn{1}{c|}{\bf Detalles de los cambios
        realizados}  &
        \bf Fecha
        \\ \hline
        V1.0         & Creación del documento
                     & 04/06/2022
        \\
        \hline
    \end{tabularx}
\end{table}

\pagebreak

\section{Introducción}
En el presente documento se detallarán todos los aspectos relacionados con la
especificación del Master Test Plan referentes al desarrollo del analizador
modular de ambiente, cuyo propósito es realizar una valoración del ambiente con
niveles de temperatura, gases presentes y ruido. Todos estos datos serán
visualizados por el usuario mediante una aplicación para smartphones.
\subsection{Contenidos}
Los contenidos del Master Test Plan son:
\begin{itemize}
    \item Asignaciones.
    \item Bases del test.
    \item Estrategia general del test.
    \item Estrategia por nivel de prueba
\end{itemize}

\section{Asignaciones}

\subsection{Responsable}
El responsable de la elaboración del documento es Luis Storaccio, encargado del
desarrollo del proyecto.
\subsection{Contratista}
La asignación es ejecutada bajo la responsabilidad de Luis Storaccio, jefe del
testing del desarrollo del proyecto.
\subsection{Alcance}
El alcance del test de aceptación es el dispositivo analizador modular de
ambiente.
\subsection{Objetivos}
Los objetivos son:
\begin{itemize}
    \item Determinar si el sistema cumple con los requerimientos.
    \item Reportar las diferencias entre comportamiento deseado y observado.
    \item Proveer de funciones de test de forma automática para el firmware,
          que
          pueda ser reutilizada en versiones futuras.
\end{itemize}
\subsection{Precondiciones}
Se realizarán formaciones para las pruebas que lo requieran.
\section{Bases del test}
Las bases del test consisten en:
\begin{itemize}
    \item Planificación y especificaciones del dispositivo.
    \item Especificación de requerimientos del software.
\end{itemize}
\section{Estrategia general del test}
\subsection{Características de calidad}
La siguiente tabla muestra la selección de las características de calidad
ISO/IEC9126 con su respectiva importancia relativa.

\begin{table}[ht]
    \centering
    \begin{tabular}{|l|l|}\hline \hline
    \rowcolor[HTML]{d6c6c3}
        \bf Características de calidad & \bf Importancia relativa (\%) \\
        \hline
        \hline
        Funcionalidad                  & 20                            \\
        Confiabilidad                  & 30                            \\
        Usabilidad                     & 30                            \\
        Eficiencia                     &                               \\
        Mantenibilidad                 & 20                            \\
        Portabilidad                   &                               \\
        \hline
    \end{tabular}
    \caption{Características de calidad vs Importancia relativa.}
\end{table}
\begin{itemize}
    \item Funcionalidad: el software del producto debe cumplir con los
          requerimientos establecidos y es por ello que posee una importancia
          relativa
          del 20\%.

    \item Confiabilidad: es prescindible que el software presente un nivel de
          confiabilidad medio, debido a que el mismo realiza la lectura de
          valores de
          gases presente en el ambiente.
    \item Usabilidad: como el producto está enfocado para usuarios
          domiciliares, el software deberá presentar al usuario un nivel bajo de
          dificultad en cuanto al conexionado y al acceso de la información. Por tal
          motivo se selecciona un nivel medio de importancia relativa.
    \item Mantenibilidad: como el producto posee conexión con internet, el
          software deberá poder actualizarse por dicho medio, actualizando sus funciones
          y eliminando posibles bugs.
\end{itemize}
\subsection{Asignación de características de calidad a los niveles de prueba}

\begin{table}[ht]
    \centering

       \begin{tabularx}{\linewidth}{@{}|c|X|c|X|c|X|c|X|c|X|c|@{}}\hline \hline
            \rowcolor[HTML]{d6c6c3}     
                   & \bf Funcionalidad  & \bf Confiabilidad & \bf Usabilidad & \bf
            Eficiencia                      & \bf Mantenibilidad & \bf Portabilidad                              \\ \hline
            \hline
            Importancia relativa            &          20        &            30     &       30       &     & 20 &
            \\
            Test Unitario                   &           ++       &           ++      &                &     &  & \\
            Test de integración de software &                    &                   &        ++        &     & + &
            \\
            Test de integración HW/SW       &           ++       &          ++       &                &     &  &
            \\
            Test de sistema                 &                    &           ++        &                &     &  & \\
            Test de aceptación              &           ++       &                   &                &     & 
             &
            \\
            Pruebas de campo                &          ++        &                 &        +        &     &  &
            \\ \hline
        \end{tabularx}
        \caption{Asignación de las características de calidad a los niveles de prueba. (++: El testeo de la característica de calidad
se realizará a fondo en este nivel de prueba); (+: El testeo de la característica de calidad será cubierto en este nivel de
prueba); (vacío: La característica de calidad no representa un problema en este nivel de prueba).}
  
    
\end{table}
\begin{itemize}
\item Funcionalidad: para verificar que se cumple con los requerimientos funcionales, se realizarán los testeos unitarios sobre cada una de las funciones con pruebas de aceptación con el cliente. Además se realizarán pruebas a nivel de integración de HW/SW y de campo.
\item Confiablidad: con el fin de verificar que el software es confiable se realizarán exhaustivas pruebas unitarias de cada una de las partes que componen lo componen. Además se realizarán testeos a nivel de sistema completo y en campo.
\item Usabilidad: se verificará la usabilidad del software realizando pruebas en nivel de integración de software con el cliente.
\item Mantenibilidad: se realizarán pruebas de mantenibilidad del software en la etapa de integración del software.
\end{itemize}

\section{División del sistema en subsistemas}

Los subsistemas que conformar al dispositivo a desarrollar son:
\begin{itemize}
\item COM: bloque de comunicación entre la APP con módulo WiFi y con el microcontrolador.
\item OUTCTRL: salida de control para indicador luminoso LED RGB y buzzer.
\item READSR: lectura de sensores de humedad, temperatura y niveles de C02.
\item TOTALSYS: total del sistema.
\end{itemize}


Determinación de las características de test a utilizar:
\begin{itemize}
\item STT: state transition testing.
\item CFT: control flow test.
\item ECT elementary comparison test.
\item CTM classification-tree method.
\end{itemize}

Nota:\\
++: El testeo de la característica de calidad se realizará a fondo en este subsistema.\\
+: el testeo de la característica calidad será cubierto en este subsistema.
Vacio: la característica calidad no representa un problema en este subsistema. \\

\subsection{Test unitario}

\begin{table}[ht]
    \centering
    \begin{tabular}{|l|l|l|l|l|l|}\hline \hline
    \rowcolor[HTML]{d6c6c3}
 Importancia relativa (\%)& &  COM & OUTCTRL & READSR & TOTALSYS\\
        \hline
       
           &  &35 & 10 &35 & 20                          \\
        Funcionalidad    & 20 & ++ & ++ & ++ & +		\\
        Confiabilidad    & 30 &+ &++ & &+                        \\
        Usabilidad       & 30 &++ &+ &++ &+                        \\
        Mantenibilidad   & 20 &+ & &+ &+                        \\
        \hline
    \end{tabular}
    \caption{Importancia de testeo por combinación de subsistema/ característic de calidad.}
\end{table}


\begin{table}[ht]
    \centering
    \begin{tabular}{|l|l|l|l|l|}\hline \hline
    \rowcolor[HTML]{d6c6c3}
 Técnica de testeo aplicada & COM & OUTCTRL & READSR & TOTALSYS\\
        \hline
       STT & & & & ++ \\
      CFT & ++ & & ++& \\
      ECT & & ++ & &  \\
        \hline
    \end{tabular}
    \caption{Técnica de testeo utilizada por cada subsistema.}
\end{table}


\subsection{Test de integración de software}
\begin{table}[ht]
    \centering
    \begin{tabular}{|l|l|l|l|l|l|}\hline \hline
    \rowcolor[HTML]{d6c6c3}
 Importancia relativa (\%)& &  COM & OUTCTRL & READSR & TOTALSYS\\
        \hline
       
            &  &35 & 10 &35 & 20                          \\
        Funcionalidad    & 20 & ++ & + & ++ & +		\\
        Confiabilidad    & 30 & ++ & ++ & & +                        \\
        Usabilidad       & 30 & ++ & + & ++ & +                        \\
        Mantenibilidad   & 20 & + & & + & +                        \\
        \hline
    \end{tabular}
    \caption{Importancia de testeo por combinación de subsistema/ característic de calidad.}
\end{table}


\begin{table}[ht]
    \centering
    \begin{tabularx}{\linewidth}{@{}|c|X|c|X|c|X|c|@{}}\hline \hline
    \rowcolor[HTML]{d6c6c3}
 Técnica de testeo aplicada & COM & OUTCTRL & READSR & TOTALSYS\\
        \hline
       STT & ++ & & & ++ \\
      CFT & & ++ & + & \\
      ECT & & + & &  \\
        \hline
    \end{tabularx}
    \caption{Técnica de testeo utilizada por cada subsistema.}
\end{table}

\subsection{Test de integración de software/hardware}
\begin{table}[ht]
    \centering
    \begin{tabularx}{\linewidth}{@{}|c|X|c|X|c|X|c|@{}}\hline \hline
    \rowcolor[HTML]{d6c6c3}
 Importancia relativa (\%)& &  COM & OUTCTRL & READSR & TOTALSYS\\
        \hline
       
         &  &35 & 10 &35 & 20                          \\
        Funcionalidad    & 20 & ++ & ++ & ++ & +		\\
        Confiabilidad    & 30 & ++ & ++ & +& +                        \\
        Usabilidad       & 30 & ++ & + & ++ & ++                        \\
        Mantenibilidad   & 20 & + & + & + & +                        \\
        \hline
    \end{tabularx}
    \caption{Importancia de testeo por combinación de subsistema/ característic de calidad.}
\end{table}


\begin{table}[ht]
    \centering
    \begin{tabularx}{\linewidth}{@{}|c|X|c|X|c|X|c|@{}}\hline \hline
    \rowcolor[HTML]{d6c6c3}
 Técnica de testeo aplicada & COM & OUTCTRL & READSR & TOTALSYS\\
        \hline
       CTM & ++ & ++ & + & ++ \\
        \hline
    \end{tabularx}
    \caption{Técnica de testeo utilizada por cada subsistema.}
\end{table}


\subsection{Test de sistema}

\begin{table}[ht]
    \centering
    \begin{tabularx}{\linewidth}{@{}|c|X|c|X|c|X|c|@{}}\hline \hline
    \rowcolor[HTML]{d6c6c3}
 Importancia relativa (\%)& &  COM & OUTCTRL & READSR & TOTALSYS\\
        \hline
       
            &  &35 & 10 &35 & 20                          \\
        Funcionalidad    & 20 & ++ & + & ++ & +		\\
        Confiabilidad    & 30 & ++ & ++ & & +                        \\
        Usabilidad       & 30 & ++ & + & ++ & +                        \\
        Mantenibilidad   & 20 & + & &  & +                        \\
        \hline
    \end{tabularx}
    \caption{Importancia de testeo por combinación de subsistema/ característic de calidad.}
\end{table}


\begin{table}[ht]
    \centering
    \begin{tabularx}{\linewidth}{@{}|c|X|c|X|c|X|c|@{}}\hline \hline
    \rowcolor[HTML]{d6c6c3}
 Técnica de testeo aplicada & COM & OUTCTRL & READSR & TOTALSYS\\
        \hline
      CTM & ++ &++ &+ & ++ \\
        \hline
    \end{tabularx}
    \caption{Técnica de testeo utilizada por cada subsistema.}
\end{table}

\subsection{Test de aceptación}

\begin{table}[ht]
    \centering
    \begin{tabularx}{\linewidth}{@{}|c|X|c|X|c|X|c|@{}}\hline \hline
    \rowcolor[HTML]{d6c6c3}
 Importancia relativa (\%)& &  COM & OUTCTRL & READSR & TOTALSYS\\
        \hline
       
            &  &35 & 10 &35 & 20                          \\
        Funcionalidad    & 20 & + & + & + & ++		\\
        Confiabilidad    & 30 & + & + & + & ++                        \\
        Usabilidad       & 30 & + & + & + & ++                        \\
        Mantenibilidad   & 20 &  & &  & ++                        \\
        \hline
    \end{tabularx}
    \caption{Importancia de testeo por combinación de subsistema/ característic de calidad.}
\end{table}


\begin{table}[ht]
    \centering
    \begin{tabularx}{\linewidth}{@{}|c|X|c|X|c|X|c|@{}}\hline \hline
    \rowcolor[HTML]{d6c6c3}
 Técnica de testeo aplicada & COM & OUTCTRL & READSR & TOTALSYS\\
        \hline
      CTM & ++ &++ &+ & ++ \\
        \hline
    \end{tabularx}
    \caption{Técnica de testeo utilizada por cada subsistema.}
\end{table}

\subsection{Test de campos}
\begin{table}[ht]
    \centering
    \begin{tabularx}{\linewidth}{@{}|c|X|c|X|c|X|c|@{}}\hline \hline
    \rowcolor[HTML]{d6c6c3}
 Importancia relativa (\%)& &  COM & OUTCTRL & READSR & TOTALSYS\\
        \hline
       
            &  &35 & 10 &35 & 20                          \\
        Funcionalidad    & 20 & ++ & ++ & ++ & ++		\\
        Confiabilidad    & 30 & + & + & + & ++                        \\
        Usabilidad       & 30 & + & + & + & ++                        \\
        Mantenibilidad   & 20 &  & &  &                         \\
        \hline
    \end{tabularx}
    \caption{Importancia de testeo por combinación de subsistema/ característic de calidad.}
\end{table}


\begin{table}[ht]
    \centering
    \begin{tabularx}{\linewidth}{@{}|c|X|c|X|c|X|c|@{}}\hline \hline
    \rowcolor[HTML]{d6c6c3}
 Técnica de testeo aplicada & COM & OUTCTRL & READSR & TOTALSYS\\
        \hline
      CTM & ++ &++ &+ & ++ \\
        \hline
    \end{tabularx}
    \caption{Técnica de testeo utilizada por cada subsistema.}
\end{table}


\end{document}